\documentclass[a4paper]{article}

\usepackage[utf8]{inputenc}
\usepackage[danish]{babel}

\usepackage{amssymb}
\usepackage{amsmath}
\usepackage{listings}

\usepackage{booktabs}

\usepackage{mathpazo}
\usepackage{euler}

\title{Styresystmer og multiprogrammering\\ Opgave G1}
\author{Emil Fabricius Svansø \and Bjørn Uhre Arnholtz}

\begin{document}

\maketitle

\section*{Opgave 1}

Funktionen {\tt insert()} går rekursivt gennem træet, den vælger det venstre
deltræ hvis {\tt data} er større en den værdi der skal indsættes og til
højre hvis den er mindre. Hvis den finder et tomt træ bruger vi {\tt malloc()}
til at allokere lager til et nyt element og ændrer pegeren til at pege på det
nye element.

{\tt print\_inorder()} kalder først {\tt print\_inorder()} på det venstre deltræ,
derefter printer den {\tt data}, hvorefter den kalder {\tt print\_inorder()} på
det højre  deltræ.

{\tt size()} fungere meget ligesom {\tt print\_inorder}, men den summer
resultaterne fra det rekursive kald.t

I {\tt to\_array} allokere vi først plads i lageret til at indeholde antallet
af elementer i træet gange størrelsen af en {\tt int}. Derefter bruger
funktionen {\tt array\_insert} til rekursivt at indsætte elementer i arrayet.
{\tt array\_insert()} tager en peger til en peger til det første element i
arrayet. Denne peger bliver inkrementeret hvergang der bliver tilføjet et
element til arrayet, på denne måde bliver elementerne indsat i arrayet i
rækkefølge.

\section*{Opgave 2}

For at lave træet om til en list definerer vi først funktionen {\tt cons} det
hægter to lister sammen. I {\tt tree2dlist} allokere vi først lager til en ny
node.

Hvis der finder et venstre undertræ kalder laver dette om til en list og
hægter det på {\tt p}, hvis det ikke findes sættter vi {\tt ls} til {\tt p},
da funktionen skal returnere det første element i listen. Hvis der findes et
højre undertræ, laver vi også dette om til en liste og hægter denne på {\tt
p}.

\section*{Opgave 3}

{\tt insert2} er implementeret på samme måde som {\tt insert}, men vi bruger
funktionen {\tt comp} til at sammenligne istedet for {\tt >}.


\end{document}
